\chapter{Previous Work}
\label{chap:related_work}
There have been several techniques developed for finding parallel data in a
wide range of multilingual corpora. 
Here we will review previous work in this area in
order to compare with the work we will present in later chapters. We will start
with the earliest work---the alignment of manually translated corpora.
\adlbr{Rather than think of this as a compendium of related work, 
it might be useful to think of a more goal-oriented chapter: setting
the stage for the rest of your dissertation. What did we know or
not know at the outset?}


% Sub-sentence alignment?
\remove{
\section{Temporary: Summaries}
This is a temporary section containing summaries of related papers. These will
be integrated into the above sections.

\citet{Dagan93}:
\citet{Chen00}: (PTMiner) Nothing special done for aligning the sentences in the 
parallel document pairs, only a method for finding document pairs. The method of
\citet{Simard93} was used for sentence alignment.
\citet{Shi06}: (DOM alignment)
\citet{Abdul-Rauf09}:
\citet{Ambati10}:
\citet{Ture11}:
\citet{Ture12}:

% Fragment alignment (a few missing here, need Munteanu07 citation)
\subsection{Parallel Fragments}
In addition to finding full parallel sentences, some researchers have looked for
parallel fragments within sentence pairs.
\citet{Munteanu06}: Starts with standard methods for finding candidate sentence
pairs. Then, computes log-likelihood ratio based scores for pairs of words, and
does a greedy word alignment based on these scores. (sliding window desc.)
\citet{Quirk07}:
\citet{Riesa12}:

\subsection{MT with Multiple Data Sources}

\subsection{Maxent Modeling}
There are a few things related to Maxent modeling that are relevant to
the supervised alignment chapter.

1. Should include some basic Maxent modeling information/citations
2. A citation on training set imbalance, or training/test mismatch
3. Feature binning/regularization: regularization over feature networks
}
