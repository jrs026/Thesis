\chapter{Related Work}
\label{chap:related_work}

Research in statistical machine translation (SMT) began as large parallel
corpora became available. These corpora include the Canadian Hansards
(French-English parliament proceedings) and the Hong Kong Laws Corpus, among many
others. While these corpora were parallel in the sense that they were created by
directly translating text in one language, they were not sentence aligned. Noise
in the form of missing data or sentences without a $1:1$ correspondence made
alignment a non-trivial problem. This lead to the development of several
approaches for aligning parallel corpora in the early 1990s. We will give an
overview of these approaches in Section \ref{sec:parallel_related}.

In addition to aligning parallel texts, there has also been a considerable
amount of work done on finding parallel sentence pairs in comparable corpora. A
comparable corpus is a multilingual collection of documents which may contain
parallel sentences, but is not completely parallel. This broad definition
includes both weakly aligned data such as timestamped multilingual news feeds,
and Wikipedia articles linked at the document level. Depending on the type of
comparable corpus, different methods may be more or less effective for finding
parallel sentences. We will split our review of comparable corpora mining
methods into two categories. In Section \ref{sec:noisy_related}, we will examine
methods used on closely aligned comparable corpora, and in Section
\ref{sec:nonnoisy_related} we will review work on extracting parallel sentences
from less related multilingual documents.

\section{Parallel Corpus Alignment}
\label{sec:parallel_related}
Perhaps the most well known work on parallel corpus alignment is \citet{Gale93}.
The authors described a sentence alignment method based on dynamic programming
which used only sentence length to determine whether or not two sentences were
parallel. This method is widely applicable since assumes almost no linguistic
knowledge.\footnote{The only bit of information about the language pair required
is a ratio of sentence lengths in characters.} Despite this, it achieves very
high accuracy on a corpus of economic reports from the Union Bank of Switzerland
in English, French and German.

The sentence alignment approach of \citet{Kay93} also used little linguistic
knowledge, though they build a bilingual dictionary from the parallel text to
facilitate alignment. \citet{Chen93} had a similar approach, except he
incorporated the learning of both sentence and word alignments into a
probabilistic model.

\section{Comparable Corpus Mining}
\label{sec:comparable_related}

\subsection{Noisy Parallel Corpora}
\label{sec:noisy_related}
The first category of work on comparable corpora mining that we will review is
on noisy parallel data. While even corpora called ``parallel'' contain some
noise, we are refering to corpora which the methods in Section
\ref{sec:parallel_related} would fail on.

\subsection{Comparable Corpora}
\label{sec:nonnoisy_related}
