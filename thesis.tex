\documentclass[12pt,oneside,final]{thesis}
\usepackage{latexsym}
\usepackage{amsmath,amsfonts}
\usepackage{graphicx}
\usepackage{fixltx2e}
\usepackage{wrapfig}
\usepackage{fancyhdr}    % Use nice looking headers along with the required footer page numbers
\usepackage{xspace}
\usepackage{url}
\usepackage{rotating} % for rotating column header titles
\usepackage{multirow} % for multirow cells in tables
\usepackage{hhline} % allows cleaner double horizontal lines (where the vertical lines cut through)
\usepackage[round]{natbib} % allows citet and citep

%\usepackage[colorlinks,plainpages=false]{hyperref} % for links to figures, tables, sections, and references (from Zhifei/Arnab)
                                                  %%% (comment out above line for submission as it must be all black
                                                  %%%  ALSO comment out the spacing adjustment below)

% Needed for the bibliography
\def\newblock{\hskip .11em plus .33em minus .07em} 
%Define the header/footer style
\pagestyle{fancy}
\fancyhf{}
\setlength{\headheight}{15pt}
\lhead{\leftmark}
\cfoot{\thepage}
\renewcommand{\headrulewidth}{0pt}
\fancypagestyle{plain}{% Redefine ``plain'' style for chapter boundaries
\fancyhf{} % clear all header and footer fields
\fancyfoot[C]{\thepage} % except the center
\renewcommand{\headrulewidth}{0pt}
\renewcommand{\footrulewidth}{0pt}}

\newcommand{\argmax}[1]{\underset{#1}{\operatorname{argmax}}}

\usepackage{color}

\newcommand{\remove}[1]{}
\newcommand{\Note}[1]{}
%\newcommand{\Note}[1]{\textbf{\large\textcolor{blue}{[#1]}}}
\newcommand{\NoteJS}[1]{\Note{#1~--JS}}

\begin{document}

\title{Parallel Sentence Discovery for Low-Resource Languages}

\author{Jason R. Smith}
\degreemonth{May}
\degreeyear{2013}
\dissertation
\doctorphilosophy
\copyrightnotice

\maketitle

%\begin{abstract}
%One of the most important factors in the performance of a statistical machine
%translation (SMT) system is the amount and quality of parallel data it is
%trained on. This has motivated work in which parallel sentences are extracted
%from document pairs that are not necessarily translations of each other.
%Collections of such semi-parallel documents are called comparable corpora. In
%this work, we aim to extract parallel sentences from comparable corpora using a
%minimal amount of supervision.
%\end{abstract}

\setcounter{chapter}{0}
\chapter{Introduction}
\label{chap:intro}

Statistical machine translation (SMT) systems are trained using a large collection of translated
sentence pairs known as a parallel corpus. Common sources of parallel data include
parliament proceedings, books, and news articles.
For some language pairs, we have large amounts of this data. For example, the Canadian
Hansards are parliamentary proceedings that give us millions of words of
French/English parallel data \citep{Germann01}. Similarly, the proceedings of
European Union parliament are a source of parallel data for all of the languages
of its member states \citep{Koehn05}. Outside of these government sources, we
also have large collections of parallel data from news agencies for some
language pairs, such as Chinese/English \citep{Ma05}.
However, for most language pairs, we have little to no data available.
In addition, even when parallel
data is available, it often does not match the domain of the data you wish to
translate, which hurts translation quality \citep{Munteanu05}.
\adlbr{This section needs some examples to make it crystal clear what you're
talking about, even for a non-NLP person. Show an example in news or government, 
and one of your other domains (wikipedia, travel, etc.)} 

The creation of new parallel corpora can be expensive, especially when bilingual
speakers are rare for the language pair of interest. \citet{Germann01a}
investigated the costs of collecting enough data to build Tamil/English SMT
system. They found that professionally translated data would cost $\$0.36$ per
word. \citet{Germann01a} and others \citep{Zaidan11} were able to reduce the
cost of creating parallel corpora by looking to non-professional translators,
but the cost is still around $\$0.10$ per word.
In order to acquire more parallel data without this cost,
researchers have looked to multilingual corpora which share some content across languages,
but are not directly translated. Such corpora are referred to as comparable
corpora, and examples include multilingual news feeds \citep{Munteanu05},
Wikipedia articles \citep{Adafre06,Smith10}, and the Web
\citep{Resnik99,Nie99,Chen00}. 
%Most work in extracting parallel sentences from
%these corpora assumes an initial bilingual dictionary or an existing parallel
%corpus.

Comparable corpora is a broad term---\citet{Fung04a} give a more
fine-grained categorization of multilingual corpora:
%TODO: Make sure this is quoted properly
\begin{enumerate}
\item Parallel corpus: A sentence-aligned corpus containing bilingual
translations of the same document. (Curated parallel corpora)
\item Noisy parallel corpus: A corpus containing non-aligned sentences that are
nevertheless mostly bilingual translations of the same document. (Hansards,
Europarl, most ``parallel'' corpora)
\item Comparable corpus: A corpus of non-aligned and non-translated documents
which are topic-aligned. (Wikipedia)
\item Quasi-comparable corpus: A multilingual corpus which is not
sentence-aligned, translated, or topic-aligned. (the Web, multilingual news feeds)
\adlbr{I don't even know what this means. Does it serve your larger point to mention this?}
\NoteJS{The purpose of these categories is just to show that very different
corpora are called comparable, and they require very different methods to mine
for parallel data. I added examples to each category, but I'm not sure if that
helps with the confusion.}
\end{enumerate}
As comparable corpora vary greatly in their structure, different methods for finding
parallel sentences are used in each.

Even corpora which are generally considered as parallel require some amount of
processing to find parallel sentences. 
A translator may chose to translate a compound sentence as two sentences, or
vice-versa, so naively assuming that sentences are aligned in order will not
work.
Also, there may be large insertions or deletions of sentences even in curated
sources of parallel data, such as the Canadian Hansards \citep{Gale93,Chen93}.
Sentence-aligning these corpora does not require existing parallel data or a
bilingual dictionary for the language pair of interest. Instead, the structure
of the documents and the lengths of the sentences are used to determine the
sentence alignment. For comparable corpora which are topic-aligned but not
directly translated, lexical information must be used to determine which
sentence pairs should be aligned \citep{Munteanu05}. When comparable corpora are
not topic-aligned, other signals are exploited to find plausible document
alignments \citep{Resnik03}.

We will examine a representative set of comparable corpora: the Web, Twitter,
and Wikipedia; describe the different signals used to identify parallel data,
and demonstrate how extracted parallel data from these corpora improve SMT
performance across several language pairs and domains. First, we scale up
previous Web mining methods \citep{Resnik03} to several terabytes of data. We
also present a novel mining approach for Twitter, making use of metadata unique
to the microblogging medium. Finally, we introduce a new sentence alignment
model for mining parallel data from Wikipedia which takes advantage of its
fine-grained topic alignment.\adlbr{I am not sure about this particular order.
Let's discuss soon.}

\section{What counts as parallel?}
This work is centered around finding parallel data---bilingual sentence pairs
which convey the same meaning. Unfortunately, it is extremely difficult, if not
impossible, to determine whether or not two sentences in different languages
have the same meaning. One language may contain gender markings that the other
does not, or the connotation of a word may be difficult to express in another
language. Examples of this problem are explored in depth by \citet{Kay97}. 
Even ignoring the cross-lingual issues, comparing the meaning of two sentences
in the same language is still quite difficult---SMT evaluation metrics
\citep{Papineni02,Banerjee05,Snover06} must address this problem.
\adlbr{I like this, but note that this stands somewhat in opposition to the Fung
paper you talk about above. This gives a very functional description of parallelism:
it's anything that makes the BLEU score go up. Of course your evaluations might
be somewhat insensitive to errors in detecting parallelism---we don't know if
they care more about precision or recall. I think you need to address this, 
probably empirically.}
\NoteJS{I don't understand how this is in conflict with the Fung paper. I only
quote that to talk about different types of comparable corpora.}

When evaluating methods for finding parallel data, we can either measure
intrinsic or extrinsic performance. Intrinsic evaluation directly measures the
quantity and quality of parallel data we extract, while extrinsic evaluation is
only concerned with how the new parallel data improves SMT performance.
In order to perform intrinsic evaluation, we need some criteria for determining whether or not a bilingual
sentence pair is parallel. This is easy if we use parallel data, but
it is preferable to evaluate our methods on the same corpora that we are
extracting data from. When designing the criteria for judging parallel
sentences, we focus on our extrinsic goal: improving SMT performance. If a sentence pair
is likely to improve performance when added to our SMT system's training data,
we would like to extract it. The details of our annotation criteria can be found
in Chapter \ref{chap:supervised}, but in all cases they are motivated by SMT
performance. To understand what will influence performance, we need to
understand modern SMT systems.

\section{Statistical Machine Translation}
While machine translation has been around in some form for many decades
\citep{Locke55}, statistical machine translation began with the work of
\citet{Brown88,Brown90,Brown93}. SMT systems have evolved since then, most notably moving from
word-based systems to phrase-based \citep{Koehn03}. Several newer systems have
been developed, focusing mostly on incorporating syntax into the translation
model \citep{Chiang05,Quirk05,Liu06,Galley06}. These systems all share some key
characteristics in how they use parallel data:

\begin{enumerate}
\item A large collection of parallel sentences are used as training data.
\item For each parallel sentence, word-to-word correspondences are found. This
step is called word-alignment, and it is usually done with unsupervised methods
\citep{Brown93,Vogel96}.
\item Pairs of phrases, or other multi-word units, are extracted from the
word-aligned sentence pairs to form a translation model.\adlbr{Example.}
\item A language model is created from large amounts of monolingual data in the
``target'' language (the language which text is translated into). This includes
the target side of the parallel training data.
\end{enumerate}

There are additional details in each model, but the main effects of adding new
parallel data are additional inputs to the translation and language models.
\adlbr{The language model argument is weaker here.}

\section{Evaluation Pipeline}
Our evaluation setup is identical across chapters---we start with {\em initial}
data that includes some standard parallel and monolingual corpora commonly used for 
translation. We also have {\em extracted} parallel data that we find in a comparable
corpus. Table \ref{tab:exp_setup} describes how we use this data to measure SMT
improvements:

\begin{table}
\begin{center}
\begin{tabular}{|c||c|c|}
\hline
& Parallel & Monolingual \\
\hline
Baseline & Initial & Initial + Extracted \\
\hline
Experimental & Initial + Extracted & Initial + Extracted \\
\hline
\end{tabular}
\end{center}
\label{tab:exp_setup}
\caption{Parallel and monolingual data used in our SMT experiments.}\adlbr{This table seems a little suspect. Why would you use initial+extracted for the baseline results? Realistically, you could have a baseline built from a large monolingual text of any kind (Maybe even in-domain). The line you list as baseline here makes sense to tease apart whether the differences come from TM or LM (that's good), but should probably include an additional baseline.}
\NoteJS{Another baseline without the extracted target side in the language model
wouldn't hurt, but I don't think it's crucial and it's not available for the
Wikipedia experiments. Everything I've done so far has used this setup. It might be something I include in some experiments.}
\end{table}

In both the baseline and experimental conditions, we include the target side of
the extracted parallel sentences in the monolingual training data. We do this to
ensure that any increase in performance is coming from the parallel data. It
would be simple to add monolingual text from a comparable corpus to an SMT
system.\adlbr{This is probably also a good baseline.}

In all experiments, the BLEU metric \citep{Papineni02} is used to evaluate SMT
performance. The BLEU metric combines $n$-gram precision (the percentage of
$n$-grams in the hypothesis translation which are found in the reference) with a
brevity penalty.
The initial data, test sets, and other details vary by experiment.


\section{Sentence Alignment}\adlbr{I agree that this probably does not fit here. The question is: what does? I think you might want to say something about prior art here, or otherwise fit in relevant parts of your lit review. Logically, a reader that has arrived at this point in your dissertation should understand the problem you're trying to solve at a high level, and might wonder what other approaches have been taken.}
In this section, we will describe our task and notation.
We will view both parallel corpora alignment and the extraction of parallel
sentences from comparable corpora as an alignment task. In either type of
alignment we are given a set of bilingual document pairs in {\em source} and {\em
target} languages. When performing parallel corpora alignment, these document
pairs will correspond to each other very strongly, while in the case of
comparable corpora, some these document pairs may contain no parallel sentences.
\citet{Munteanu05} take their document pairs from news stories published at
roughly the same time, while \citet{Adafre06,Smith10} use entries from
Wikipedia that are on the same topic (Figure \ref{fig:wiki} gives and example).
The task of finding comparable document pairs is not addressed in this work.

\begin{figure*}[ht]
\includegraphics[width=\textwidth]{images/wiki.jpg}
\caption{An example of a Spanish/English document pair from Wikipedia.\adlbr{I think you want this example to appear in your comparable corpus section.}}
\label{fig:wiki}
\end{figure*}

Each document pair contains a sequence of source sentences (denoted by ${\bf
S}$) and target sentences (denoted by ${\bf T}$). Individual source and target
sentences are referred to by $S$ and $T$ respectively. Similarly, we refer to
the words within source and target sentences with the lowercase $s$ and $t$. We
borrow the notation of \citep{Och03} for describing alignments between sentences
as subsets of the Cartesian product of sentence positions. Sentence alignments
are referred to with the uppercase $A$, and word alignments with the lowercase
$a$.

The goal of sentence alignment is to identify which sentence pairs in the
bilingual document pairs are parallel. We view this as a retrieval task for
parallel sentence pairs, and so when annotated sentence alignments are present,
we can compute precision, recall, and F-measure.


\setcounter{chapter}{1}
\chapter{Discriminative Sentence Alignment}
\label{chap:supervised}

In this chapter we will describe a discriminative models for performing sentence
alignment on comparable document pairs. We use Wikipedia as our first source for
comparable documents, and use a conditional random field \citep{Lafferty01} as
the sentence alignment model. We also apply a discriminative monotonic alignment model to
comparable documents mined from the Web as described by \citet{Uszkoreit10}.

\section{Related Work}
\label{sec:comparable_related}

In addition to aligning parallel texts, there has also been a considerable
amount of work done on finding parallel sentence pairs in comparable corpora. A
comparable corpus is a multilingual collection of documents which may contain
parallel sentences, but is not completely parallel.\adlbr{This seems repetetive, given intro.} 
This broad definition
includes both weakly aligned data such as timestamped multilingual news feeds,
and Wikipedia articles linked at the document level. Depending on the type of
comparable corpus, different methods may be more or less effective for finding
parallel sentences. We will split our review of comparable corpora mining
methods into two categories. In Section \ref{sec:noisy_related}, we will examine
methods used on closely aligned comparable corpora, and in Section
\ref{sec:nonnoisy_related} we will review work on extracting parallel sentences
from less related multilingual documents.

\subsection{Noisy Parallel Corpora}
\label{sec:noisy_related}
The first category of work on comparable corpora mining that we will review is
on noisy parallel data. While even corpora called ``parallel'' contain some
noise, we are refering to corpora which the methods in Section
\ref{sec:parallel_related} would fail on.

Similar to the dynamic programming approaches explored in Section
\ref{sec:parallel_related}, \citet{Zhao02} used a dynamic programming strategy
for aligning parallel sentences in a document pair. They create a probabilistic
model of a comparable document pair $P(S,T,A)$ and choose an alignment to
maximize the probability of the observed source and target documents. To
estimate the probability of two sentences being aligned, they used and IBM-style 
word alignment models (Model 3, specifically) which were estimated on existing
parallel data. \citet{Zhao02} also describes a bootstraping approach where high
confidence sentence alignments are added to the training data for the word
alignment model, and then sentence alignments are recomputed. Much of the work
on noisy parallel/comparable corpora mining used this technique 
\citep{Fung04a,Fung04b,Wu05,Munteanu05}.

\subsection{Finding Parallel Sentences}
Once comparable document pairs have been identified, most comparable corpora
extraction methods will independently judge each sentence pair as parallel or
non-parallel. Since there is often a very large amount of document pairs and
thus potential sentence pairs, filters are used to prune out sentence pairs that
are highly unlikely to be parallel. For example, \citet{Munteanu05} used a
sentence length filter to remove sentence pairs where one sentence was more than
twice as long as the other. In addition, they used a word overlap filter based
on the bilingual dictionary used to find candidate document pairs.

Given a filtered set of sentence pairs, more expensive methods of scoring
sentence pairs can be used. \citet{Munteanu05} use a binary MaxEnt
classifier to ultimately determine whether or not a sentence pair is parallel.
The classifier is trained on parallel data and makes used of features which are
mostly based on word alignments. Others
\cite{Fung04a,Fung04b,Tillmann09a,Tillmann09b} use a single score for sentence
pairs based on either a word alignment model or bag-of-words similarity after
projection through a bilingual lexicon, and tune a threshold on held out data.

\section{Wikipedia as a Comparable Corpus}
\label{sec:wiki}
Wikipedia \citep{wikipedia} is an online collaborative encyclopedia available in
a wide variety of languages.  While the English Wikipedia is the largest, with
over 3 million articles, there are 24 language editions with at least 100,000
articles.

Articles on the same topic in different languages are also connected via
``interwiki'' links, which are annotated by users.  This is an extremely
valuable resource when extracting parallel sentences, as the document alignment
is already provided.  
Table \ref{table:interwiki} shows how
many of these ``interwiki'' links are present between the English Wikipedia and the
16 largest non-English Wikipedias.

\begin{table*}
\small
\begin{center}
\begin{tabular}{|c|c|c|c|c|c|c|c|}
\hline
French & German & Polish & Italian & Dutch & Portuguese & Spanish & Japanese \\
496K & 488K & 384K & 380K & 357K & 323K & 311K & 252K\\
\hline
Russian & Swedish & Finnish & Chinese & Norwegian & Volap\"{u}k & Catalan & Czech \\
232K & 197K & 146K & 142K & 141K & 106K & 103K & 87K\\
\hline
\end{tabular}
\end{center}
\caption{Number of aligned bilingual articles in Wikipedia by language (paired with English).}
\label{table:interwiki}
\end{table*}

Wikipedia's markup contains other useful indicators for parallel sentence
extraction.  The many hyperlinks found in articles have previously been used as
a valuable source of information.  \citep{Adafre06} use matching
hyperlinks to identify similar sentences.  Two links match if the articles they
refer to are connected by an ``interwiki'' link.
Also, images in Wikipedia are often stored in a central source across
different languages; this allows identification of captions which may be
parallel.  Finally, there are other minor forms
of markup which may be useful for finding similar content across languages, such
as lists and section headings.  In Section \ref{sec:features}, we will explain
how features are derived from this markup.

\section{Models for Parallel Sentence Extraction}
\label{sec:models} In this section, we will focus on methods for
extracting parallel sentences from aligned, comparable documents.
The related problem of automatic document alignment in news and
web corpora has been explored by a number of researchers,
including
\citet{Resnik03}, \citet{Munteanu05},
\citet{Tillmann09a}, and \citet{Tillmann09b}.
Since our corpus already contains document alignments, we sidestep
this problem, and will not discuss further details of this issue.
That said, we believe that our methods will be effective in
corpora without document alignments when combined with one of the
aforementioned algorithms.


\subsection{Binary Classifiers and Rankers}
Much of the previous work involves building a binary classifier for sentence
pairs to determine whether or not they are parallel
\citep{Munteanu05,Tillmann09a}.
The training data usually comes from a standard parallel corpus.  There is a
substantial class imbalance ($O(n)$ positive examples, and $O(n^2)$
negative examples), and various heuristics are used to mitigate this problem.
\citet{Munteanu05} filter out negative
examples with high length difference or low word overlap (based on a bilingual
dictionary).

We propose an alternative approach: we learn a ranking model, which, for each sentence in the {\em
source} document, selects either a sentence in the {\em target} document that it is
parallel to, or ``null''.
This formulation of the problem avoids the class imbalance issue of the binary classifier.

In both the binary classifier approach and the ranking approach, we use a Maximum Entropy
classifier, following \citet{Munteanu05}.

\subsection{Sequence Models}
In Wikipedia article pairs, it is common for parallel sentences to
occur in clusters.  A global sentence alignment model is
able to capture this phenomenon. For both parallel and comparable
corpora, global sentence alignments have been used, though the
alignments were monotonic
\citep{Gale93,Moore02,Zhao02}.
Our model is a first order linear chain conditional random field
(CRF) \citep{Lafferty01}. The set of source and
target sentences are observed. For each {\em source} sentence, we
have a hidden variable indicating the corresponding {\em target}
sentence to which it is aligned (or null). The model is similar to the
discriminative CRF-based word alignment model of
\citep{Blunsom06}.

\subsection{Features}
\label{sec:features}
Our features can be grouped into four categories.

\subsubsection{Features derived from word alignments}
%kristina-edit-1

We use a feature set inspired by \citep{Munteanu05}, who defined
features primarily based on IBM Model 1 alignments
\citep{Brown93}.  We also use HMM word alignments
\citep{Vogel96} in both directions ({\em source} to {\em target} and
{\em target} to {\em source}), and extract the following features based on these
four alignments:\footnote{These are all derived from the one best alignment, and
normalized by sentence length.}

\begin{enumerate}
\item Log probability of the alignment
\item Number of aligned/unaligned words
\item Longest aligned/unaligned sequence of words
\item Number of words with fertility 1, 2, and 3+
\end{enumerate}

We also define two more features which are independent of word alignment
models.  One is a sentence length feature taken from \citep{Moore02}, which
models the length ratio between the {\em source} and {\em target} sentences with
a Poisson distribution.  The other feature is the difference in relative
document position of the two sentences, capturing the idea that the aligned
articles have a similar topic progression.

The above features are all defined on sentence pairs, and are included in the
binary classifier and ranking model.
\subsubsection{Distortion features} In the sequence model, we use additional distortion
features, which only look at the difference between the position
of the previous and current aligned sentences.  One set of
features bins these distances; another looks at the
absolute difference between the expected position (one after the
previous aligned sentence) and the actual position.

\subsubsection{Features derived from Wikipedia markup}

Three features are derived from Wikipedia's
markup. The first is the number of matching links in the sentence
pair. The links are weighted by their inverse frequency in the
document, so a link that appears often does not contribute much to
this feature's value.  The image feature fires whenever two
sentences are captions of the same image, and the list feature
fires when two sentences are both items in a list.  These last two
indicator features fire with a negative value when the feature
matches on one sentence and not the other.

None of the above features fire on a null alignment, in either the
ranker or CRF.  There is also a bias feature for these two models, which
fires on all non-null alignments.

\subsubsection{Word-level induced lexicon features}
In order to address sparsity issues in our seed parallel corpora, we introduce a
bilingual lexicon model which learns word translation probabilities from the
linked Wikipedia articles. The details of this model and the features derived
from it can be found in \citep{Smith10}.

\section{Experiments}
\label{sec:exp}

\subsection{Data}
We annotated twenty Wikipedia article pairs for three language pairs: Spanish-English,
Bulgarian-English, and German-English.
Each sentence in the {\em source} language was annotated with
possible parallel sentences in the {\em target} language (the target language was
English in all experiments).  The pairs were annotated with a quality level:
{\bf 1} if the sentences contained some parallel fragments, {\bf 2} if the sentences
were mostly parallel with some missing words, and {\bf 3} if the sentences appeared to be direct
translations.  In all experiments, sentence pairs with quality {\bf 2} or {\bf 3} were
taken as positive examples. 

\begin{table*}[ht]
\tiny
\begin{center}
\begin{tabular}{|c||c|c|c||c|c|c||c|c|c|}
\hline
Language Pair     & \multicolumn{3}{|c||}{Binary Classifier} & \multicolumn{3}{|c||}{Ranker} & \multicolumn{3}{|c|}{CRF} \\
\hline
                  & Avg Prec & R@90 & R@80
                  & Avg Prec & R@90 & R@80
                  & Avg Prec & R@90 & R@80 \\
\hline
English-Bulgarian & 75.7  & 33.9  & 56.2    & 76.3  & 38.8  & 57.0    & {\bf 80.6}  & {\bf 52.9}  & {\bf 59.5} \\
English-Spanish   & 90.4  & 81.3  & 87.6    & 93.4  & 81.0  & 84.5    & {\bf 94.7}  & {\bf 87.6}  & {\bf 90.2} \\
English-German    & 61.8  &  9.4  & 27.5    & 66.4  & 25.7  & 42.4    & {\bf 78.9}  & {\bf 52.2}  & {\bf 54.7} \\
\hline
\end{tabular}
\end{center}
\caption{Average precision, recall at 90\% precision, and recall at 80\%
precision for each model in all three language pairs.  In these experiments, the
Wikipedia features and lexicon features are omitted.}
\label{table:modelcompare}
\end{table*}

\begin{table*}[ht]
\small
\begin{center}
\begin{tabular}{|c||c|c|c||c|c|c|}
\hline
Setting           & \multicolumn{3}{|c||}{Ranker} & \multicolumn{3}{|c|}{CRF} \\
\hline
                  & Avg Prec & R@90 & R@80
                  & Avg Prec & R@90 & R@80\\
\hline
English-Bulgarian & & & & & & \\
\hline
One Direction          & 76.3  & 38.8  & 57.0    & 80.6  & 52.9  & 59.5 \\
Intersected            & 78.2  & 47.9  & 60.3    & 79.9  & 38.8  & 57.0 \\
Intersected +Wiki      & 80.8  & 39.7  & 68.6    & 82.1  & 53.7  & 62.8 \\
Intersected +Wiki +Lex & 89.3  & 64.4  & 79.3    & {\bf 90.9}  & {\bf 72.0}  & {\bf 81.8} \\
\hline
English-Spanish & & & & & & \\
\hline
One Direction          & 93.4  & 81.0  & 84.5    & 94.7  & 87.6  & 90.2 \\
Intersected            & 94.3  & 82.4  & 89.0    & 95.4  & 88.5  & 91.8 \\
Intersected +Wiki      & 94.5  & 82.4  & 89.0    & 95.6  & 89.2  & 92.7 \\
Intersected +Wiki +Lex & 95.8  & 87.4  & 91.1    & {\bf 96.4}  & {\bf 90.4}  & {\bf 93.7} \\
\hline
English-German & & & & & & \\
\hline
One Direction          & 66.4  & 25.7  & 42.4    & 78.9  & 52.2  & 54.7 \\
Intersected            & 71.9  & 36.2  & 43.8    & 80.9  & 54.0  & 67.0 \\
Intersected +Wiki      & 74.0  & 38.8  & 45.3    & 82.4  & 56.9  & {\bf 71.0} \\
Intersected +Wiki +Lex & 78.7  & 46.4  & 59.1    & {\bf 83.9}  & {\bf 58.7}  & 68.8 \\
\hline
\end{tabular}
\end{center}
\caption{Average precision, recall at 90\% precision, and recall at 80\%
precision for the Ranker and CRF in all three language pairs.  ``+Wiki''
indicates that Wikipedia features were used, and ``+Lex'' means the lexicon
features were used.}
\label{table:featurecompare}
\end{table*}

For our seed parallel data, we used the Europarl corpus
\citep{Koehn05} for Spanish and German and the
JRC-Aquis corpus for Bulgarian, plus the article titles for
parallel Wikipedia documents, and translations available from
Wiktionary entries.\footnote{Wiktionary is an online collaborative
dictionary, similar to Wikipedia.} %kristina-edit

\subsection{Intrinsic Evaluation}
Using 5-fold cross-validation on the 20 document pairs for each
language condition, we compared the binary classifier, ranker, and
CRF models for parallel sentence extraction. To tune for 
precision/recall, we used minimum Bayes risk decoding.  We define
the loss $L(\tau,\mu)$ of picking target sentence $\tau$ when the
correct target sentence is $\mu$ as $0$ if $\tau=\mu$, $\lambda$ if
$\tau = \textsc{null}$ and $\mu\ne\textsc{null}$, and $1$ otherwise.
By modifying the null loss $\lambda$, the precision/recall
trade-off can be adjusted.  For the CRF model, we used posterior
decoding to make the minimum risk decision rule tractable. As a
summary measure of the performance of the models at different
levels of recall we use average precision as defined in
\citep{Ido06}. We also report recall at precision of 90 and
80 percent. %kristina-edit 
 Table \ref{table:modelcompare} compares the
different models in all three language pairs.

In our next set of experiments, we looked at the effects of the Wikipedia
specific features.  Since the ranker and CRF are asymmetric models,
we also experimented with running the models in both directions and combining
their outputs by intersection.  These results are shown in Table \ref{table:featurecompare}.

Identifying the agreement between two asymmetric models is a commonly
exploited trick elsewhere in machine translation. It is mostly effective
here as well, improving all cases except for the Bulgarian-English CRF where
the regression is slight. More successful are the Wikipedia features, which
provide an auxiliary signal of potential parallelism.

The gains from adding the lexicon-based features can be dramatic
as in the case of Bulgarian (the CRF model average precision
increased by nearly 9 points). The lower gains on Spanish and
German may be due in part to the lack of language-specific
training data. These results are very promising and
motivate further exploration. We also note that this
is perhaps the first successful practical application of an
automatically induced word translation lexicon.

\subsection{SMT Evaluation}

We also present results in the context of a full machine translation system
to evaluate the potential utility of this data.  A standard
phrasal SMT system \citep{Koehn03} serves as our testbed, using
a conventional set of models: phrasal models of source given target and
target given source; lexical weighting models in both directions, language
model, word count, phrase count, distortion penalty, and a lexicalized
reordering model.  Given that the extracted Wikipedia data takes the
standard form of parallel sentences, it would be easy to exploit this same
data in a number of systems.

\begin{table*}[ht]
\tiny
\begin{center}
\begin{tabular}{|rr||r|r||r|r||r|r|}
\hline
      &                & German        & English       & Spanish       & English      & Bulgarian    & English   \\
\hline
      & sentences     & 924,416       & 924,416       & 957,884       & 957,884      & 413,514      & 413,514   \\
\textbf{Medium} \
      & types     & 351,411       & 320,597       & 272,139       & 247,465      & 115,756      & 69,002    \\
      & tokens    & 11,556,988    & 11,751,138    & 18,229,085    & 17,184,070   & 10,207,565   & 10,422,415\\
\hline
      & sentences      & 6,693,568     & 6,693,568     & 7,727,256     & 7,727,256    & 1,459,900    & 1,459,900 \\
\textbf{Large} \
      &      types     & 1,050,832     & 875,041       & 1,024,793     & 952,161      & 239,076      & 137,227   \\
      &      tokens    & 100,456,622   & 96,035,475    & 155,626,085   & 137,559,844  & 29,741,936   & 29,889,020\\
\hline
      & sentences      & 1,694,595     & 1,694,595     & 1,914,978     & 1,914,978    & 146,465      & 146,465   \\
\textbf{Wiki}  \
      &      types     & 578,371       & 525,617       & 569,518       & 498,765      & 107,690      & 74,389    \\
      &      tokens    & 21,991,377    & 23,290,765    & 29,859,332    & 28,270,223   & 1,455,458    & 1,516,231 \\
\hline
\end{tabular}
\end{center}
\caption{Statistics of the training data size in all three language pairs.}
\label{table:mtTrainStats}
\end{table*}

\begin{table*}[ht]
\small
\begin{center}
\begin{tabular}{|rr||r|r||r|r||r|r|}
\hline
      &                & German        & English       & Spanish       & English      & Bulgarian     & English   \\
\hline
\textbf{Dev A} \
      & sentences      & 2,000         & 2,000         & 2,000         & 2,000        & 2,000         & 2,000     \\
      & tokens    & 16,367        & 16,903        & 24,571        & 21,493       & 39,796        & 40,503    \\
\hline
\textbf{Test A}\
      & sentences      & 5,000         & 5,000         & 5,000         & 5,000        & 2,473         & 2,473     \\
      & tokens    & 42,766        & 43,929        & 68,036        & 60,380       & 52,370        & 52,343    \\
\hline
\textbf{Wikitest}\
      & sentences    & 500           & 500           & 500           & 500          & 516           & 516       \\
      & tokens    & 8,235         & 9,176         & 10,446        & 9,701        & 7,300         & 7,701     \\
\hline
\end{tabular}
\end{center}
\caption{Statistics of the test data sets.}
\label{table:mtTestStats}
\end{table*}

For each language pair we explored two training conditions.  The
``Medium'' data
condition used easily downloadable corpora: Europarl for German-English and
Spanish-English, and JRC/Acquis for Bulgarian-English.  Additionally we
included titles of all linked Wikipedia articles as parallel sentences in
the medium data condition.  The ``Large'' data condition includes all the
medium
data, and also includes using a broad range of available sources such as
data scraped from the web~\citep{Resnik03}, data from the United
Nations, phrase books, software documentation, and more.

In each condition, we explored the impact of including additional
parallel sentences automatically extracted from Wikipedia in the
system training data. For German-English and Spanish-English, we
extracted data with the null loss adjusted to
achieve an estimated precision of 95 percent, and for
English-Bulgarian a precision of 90 percent. %kristina-edit-1
Table~\ref{table:mtTrainStats} summarizes the characteristics of
these data sets.  We were pleasantly surprised at the amount of
parallel sentences extracted from such a varied comparable corpus.
Apparently the average Wikipedia article contains at least a
handful of parallel sentences, suggesting this is a very fertile
ground for training MT systems.

The extracted Wikipedia data is likely to make the greatest impact on broad
domain test sets -- indeed, initial experimentation showed little BLEU gain
on in-domain test sets such as Europarl, where out-of-domain training data
is unlikely to provide appropriate phrasal translations.  Therefore, we
experimented with two broad domain test sets.

First, Bing Translator provided a sample of translation
requests along with translations in German-English and
Spanish-English -- this constituted our standard development and
test set for those language pairs.  Unfortunately no such tagged
set was available in Bulgarian-English, so we held out a portion
of the large system's training data to use for development and
test. In each language pair, the test set was split into a
development portion (``Dev A'') used for minimum error rate
training~\citep{OchMert03} and a test set (``Test A'') used
for final evaluation.

\begin{table*}[ht!]
\small
\begin{center}
\begin{tabular}{|lr||l||l|l|}
\hline
Language pair     & Training data     & Dev A             & Test A            & Wikitest     \\
\hline
Spanish-English   & Medium            & 32.6              & 30.5              & 33.0         \\
                  & Medium+Wiki       & 36.7 (+4.1)       & 33.8 (+3.3)       & 39.1 (+6.1)  \\
                  & Large             & 39.2              & \textbf{37.4}     & 38.9         \\
                  & Large+Wiki        &\textbf{39.5} (+0.3)&37.3 (-0.1)       & \textbf{41.1} (+2.2)  \\
\hline
German-English    & Medium            & 28.7              & 26.6              & 13.0         \\
                  & Medium+Wiki       & 31.5 (+2.8)       & 29.6 (+3.0)       & 18.2 (+5.2)  \\
                  & Large             & \textbf{35.0}     & 33.7              & 17.1         \\
                  & Large+Wiki        & 34.8 (-0.2)       &\textbf{33.9} (+0.2)&\textbf{20.2} (+3.1)  \\
\hline
Bulgarian-English & Medium            & 36.9              & 26.0              & 27.8         \\
                  & Medium+Wiki       & 37.9 (+1.0)       & 27.6 (+1.6)       & 37.9 (+10.1) \\
                  & Large             &\textbf{51.7}      &\textbf{49.6}      & 36.0         \\
                  & Large+Wiki        &\textbf{51.7}(+0.0)& 49.4 (-0.2)       &\textbf{39.5}(+3.5)  \\
\hline
\end{tabular}
\end{center}
\caption{BLEU scores of MT systems under various training and test
conditions.  The final BLEU score from minimum error rate training is in the
first column; two additional columns are BLEU scores on held-out test sets.
For training data conditions including the extracted Wikipedia sentences,
the parenthesized values indicate the absolute BLEU difference against the
corresponding system without Wikipedia extracts.}
\label{table:mtTestResults}
\end{table*}

Second, we created new test sets in each of the three language
pairs by sampling parallel sentences from held out Wikipedia
articles.  To ensure that this test data was clean, we manually
filtered the sentence pairs that were not truly parallel and
edited them as necessary to improve adequacy.  We called
this ``Wikitest''. Characteristics of
these test sets are summarized in Table~\ref{table:mtTestStats}.

We evaluated the resulting systems using
BLEU-4~\citep{Papineni02}; the results are presented in
Table~\ref{table:mtTestResults}.  First we note that the extracted
Wikipedia data are very helpful in medium data conditions,
significantly improving translation performance in all conditions.
Furthermore we found that the extracted Wikipedia sentences
substantially improved translation quality on held-out Wikipedia
articles. In every case, training on medium data plus Wikipedia
extracts led to equal or better translation quality than the large
system alone. Furthermore, adding the Wikipedia data to the large
data condition still made substantial improvements.

\section{Comparable documents from the Web}

Multilingual websites have often been used as a source of parallel data
\citep{Resnik03,Huang05,Shi06}. Most approaches for finding potential parallel
documents rely on metadata rather than the content of the websites. 
This metadata may include links which appear to point to alternate versions of the same
page in another language, the URLs of the websites, or their HTML structure.
Another approach for finding multilingual websites is given by
\citet{Uszkoreit10}, who translate all non-English web pages into English using
their MT system and then use monolingual document similarity measures to find
comparable document pairs. \citet{Uszkoreit10} also describe a sentence
alignment model which is applied to the comparable document pairs. In this work,
we will instead use a discriminative, monotonic alignment model to align these document
pairs. As this requires labeled sentence alignment data for both training and
evaluation, we also describe an annotation tool for monotonic alignments which 
allows $m:n$ sentence matchings.

\subsection{Data collection}

\subsubsection{Annotation Tool}

\begin{figure}
\begin{center}
\includegraphics[scale=0.5]{images/google_annotate.png}
\caption{The annotation interface for sentence alignment.
\NoteJS{Take or adapt the figure from the Google slides.}
}
\label{fig:google_annotate}
\end{center}
\end{figure}

We obtain a set of Japanese-English document pairs using the method described by
\citet{Uszkoreit10}. From this set, we randomly selected roughly 1000 document
pairs to be annotated. The annotators were presented with the interface shown in
Figure \ref{fig:google_annotate}. For each document pair we asked if
the two documents are in the correct language pair, and if one of the documents
appeared to be machine translated. If the language pair was correct and the
documents appeared to be translated by a human, then the annotators aligned the
sentences in the document.

The annotation tool is closely related to the monotonic alignment model in how
it operates. The user is asked whether or not the first sentences in each
document are parallel. If they are, the user selects ``Match'', the sentences
are aligned, and the user is then prompted about the next two sentences. If the
sentence pair presented to the annotator is not parallel, they will select
either ``Skip Left'' or ``Skip Right'', which advance the current sentence in
the document on the left or right. These three operations correspond to the
operations used in the monotonic alignment model, and the annotator's sequence
of operations can be directly used as training data.

The tool also allows the annotator to specify arbitrary $n:m$ sentence
alignments throught the ``Merge Left'' and ``Merge Right'' buttons. The merge
actions append the immediately following sentences to the currently selected
sentences, which can then be aligned.

\subsubsection{Data Collection Results}

Since the data we have annotated using this tool is taken from the HTML source
of webpages, and converted to plain text and segmented into sentences
automatically, the ``Merge'' buttons are often used to correct errors in
sentence segmentation. This results in a few large $n:m$ alignments, though most
alignments are still $1:1$. Table \ref{table:merge_stats} breaks down the
alignments that the annotators have provided.

\begin{table*}[ht!]
\begin{center}
\begin{tabular}{|l||l|l|l|l|l|}
\hline
Alignment Type & 1:1 & 2:1 & 1:2 & 2:2 & Other\\
\hline
Percentage & 95.8\% & 1.70\% & 1.28\% & 0.159\% & 1.12\%\\
\hline
\end{tabular}
\end{center}
\caption{Statistics on the types of $n:m$ alignments found in the annotated
data.}
\label{table:merge_stats}
\end{table*}

\subsection{Monotonic alignment model}

\begin{figure}
\begin{center}
\includegraphics[scale=0.5]{images/google_model.png}
\caption{The discriminative model for monotonic sentence alignment. The
highlighted path represents a possible alignment between the two documents.}
\label{fig:mono_disc}
\end{center}
\end{figure}

Our sentence alignment model is a discriminatively trained monotonic alignment
model. An illustration of this model is given in Figure \ref{fig:mono_disc}.
Formally, we represent the model as a weighted finite-state automata (WFSA) with features
that fire on each arc as described in \citet{Eisner02}. Specifically, in this
work we use a weighted finite-state transducer (WFST) to describe our alignment
model. A WFST includes a set of states
$\mathbb{Q}$, source and target alphabets $\Sigma_S$ and $\Sigma_T$,
transition function $\delta : \mathbb{Q} \times \Sigma_S \times \Sigma_T
\longrightarrow \mathbb{Q} \times W$,
and start and end states $\mathbb{S}$ and $\mathbb{F}$.

\begin{align*}
\mathbb{Q} &=& \{q_{i,j} \mid 0 \leq i \leq |\vec{S}|, 0 \leq j \leq |\vec{T}|\}\\
\Sigma_S &=& \{\epsilon\} \cup \{S_i \mid 0 \leq i < |\vec{S}|\}\\
\Sigma_T &=& \{\epsilon\} \cup \{T_j \mid 0 \leq j < |\vec{T}|\}\\
\delta(q_{i, j}) &=& \{(q_{i+1, j}, (S_i, \epsilon), \vec{f}(S_i, \epsilon)\cdot\vec{w}),\\
& & (q_{i, j+1}, (\epsilon, T_j), \vec{f}(\epsilon, T_j)\cdot\vec{w}),\\
& & (q_{i+1,j+1}, (S_i, T_j), \vec{f}(S_i, T_j)\cdot\vec{w})\}\\
\mathbb{S} &=& q_{0,0}\\
\mathbb{F} &=& q_{|\vec{S}|, |\vec{T}|}
\end{align*}

\NoteJS{The transition function still needs some work, and I need to mention
semirings. A section on notation should help with this.}

The FSM representing the set of possible alignments for a document pair 
$(\vec{S}, \vec{T})$ has $(|\vec{S}|+1)\cdot(|\vec{T}|+1)$ states. These states
correspond to positions in the source and target documents. Each state has at
most three transitions, which either skip the current source sentence, skip the
current target sentence, or align the current source and target sentences. Note
that these operations directly match the actions used by the annotators to
generate the labeled alignments, so they may be directly used as an observed
path through the WFSA. This automata only contains $1:1$ alignments, though
$n:m$ alignments could be added to this machine through additional arcs on each
state.

The weights on each of the arcs $a$ come from the dot product of the features which
fire on that arc and the weight vector $\vec{w}$. 
The total weight of a path $\pi = a_0,a_1,\dots,a_n$ is the dot product of all
features firing on each arc and the weight vector. Using this property we can
now define a probability distribution over paths through this WFST:

\begin{align*}
p(\pi|\vec{S}, \vec{T}) &=& \frac{\sum_{a \in \pi}
\exp \vec{f}(a)\cdot\vec{w}}{\sum_{\pi'}\sum_{a \in \pi'}\exp \vec{f}(a)\cdot\vec{w}}
\end{align*}

The denominator is the sum of the weights of all paths through the WFST.
We train our model to maximize the probability of the observed training data:

\begin{align*}
\argmax{\vec{w}} \sum_{(\pi,\vec{S},\vec{T})} p(\pi|\vec{S},\vec{T})
\end{align*}

The gradient of this objective can be computed using standard finite-state
algorithms \citep{Eisner02}.

\subsubsection{Features}
\label{sec:mono_feats}
We chose to use a simple set of features which are mostly based on
bag-of-words similarity measures. First, we have bias features which fire on all
``matching'' and ``mismatching'' arcs (arcs which emit $(S_i, T_j)$ are matching
arcs, and the mismatching arcs emit either $(\epsilon, T_j)$ or $(S_i,
\epsilon)$). For the matching arcs, we project the source sentence through a
weighted bilingual dictionary and compute cosine similarity with the target
sentence (and the same is done in reverse).\NoteJS{I don't have the exact set of
features used in the experiments from my Google slides.}

We also experimented with a set of first-order features, which look at the
previous arc taken in the WFST. This is made possible by splitting each state
$q_{i,j}$ into $q_{i,j,+}$ and $q_{i,j,-}$. ``Match'' transitions lead to
$q_{i,j,+}$ while ``mismatch'' transitions lead to $q_{i,j,-}$. Since the states
are now recording the last operation, the arcs leaving these states have
features which fire on two matches in a row, for example. The intuition behind
adding these features is that matches often follow other matches, and a similar
trend is present for mismatches.

\subsection{Results}

We compare our discriminatively trained model against a hand-tuned monotonic
alignment model. Our results are shown in Table \ref{table:google_results}.

\begin{table*}
\begin{center}
\begin{tabular}{|l||l|l|l|}
\hline
Model & Precision & Recall & F1\\
\hline
\hline
Baseline & 63.9\% & 95.0\% & 76.4\%\\
\hline
Discriminative Aligner & {\bf 66.0\%} & 94.4\% & {\bf 77.7}\%\\
\hline
+First Order Features & 65.1\% & {\bf 95.2\%} & 77.3\%\\
\hline
\end{tabular}
\caption{Precision, recall and F1 score measured on a held out set of aligned
Japanese-English document pairs.}
\label{table:google_results}
\end{center}
\end{table*}


\setcounter{chapter}{2}
\chapter{Unsupervised Parallel Sentence Extraction from Comparable Corpora}
\label{chap:unsupervised}

\section{Unsupervised Sentence Alignment}
\label{sec:alignment}
In most previous work on finding parallel sentences in comparable corpora, some
initial parallel data (parallel sentences or bilingual dictionary entries) is
used as a starting point. This data is used to extract parallel sentences, with
the hope that the bilingual word correspondences from the initial data are enough to
determine whether or not two sentences are parallel. The obvious drawback is
the reliance on the initial data, which may be small. Ideally, one would learn
additional word correspondences from parallel sentences that were extracted, and
this information could be used to find more parallel sentences. In fact, this
bootstrapping method has been used in previous work \citep{Fung04a,Fung04b,Wu05}.

We will explore a novel way of using semi-supervised learning to find
parallel sentences: by including sentence and word alignment in a single model.
Much like the IBM word alignment models \citep{Brown93} which can be trained on
sentence pairs without word alignment data, our model can be trained on document
pairs without sentence or word alignment data, and can similarly be trained using
the expectation-maximization (EM) algorithm \citep{Dempster77}.

\subsection{Model}

First we must define a generative model of a bilingual (possibly) parallel
document pair. We will use a joint model of the source and target documents
based on stochastic edit distance \citep{Ristad98}. Document pairs are
generated by a memoryless transducer which generates substitution pairs $(S,T)$,
insertion pairs $(\epsilon, T)$, deletion pairs $(S,\epsilon)$, and the
termination pair $(\epsilon, \epsilon)$, borrowing the convention used by
\citep{Oncina06} for simplicity. Substitution pairs correspond to parallel
source and target sentences, while the insertion and deletion pairs are
monolingually generated. For this model to be properly defined, the probability
of generating all pairs must sum to one:

\begin{equation}
\sum_{x \in S\cup \{\epsilon\}, y \in T\cup \{\epsilon\}} p(x,y) = 1
\end{equation}

Since the insertion and deletion operations are monolingual generation of
sentences, we use a standard $n$-gram language model for their probabilities.
For the probability of a substitution pair, we decompose $p(S,T)$ into
$p(T|S)p(S)$. $p(T|S)$ is defined by an IBM word alignment model \citep{Brown93}
(Model 1 in this preliminary work), and $p(S)$ is given by the same language
model used to generate deletion pairs ($(S,\epsilon)$). Since $p(S,T)$,
$p(S,\epsilon)$ and $p(\epsilon,T)$ all individually sum to one, they must be
weighted to ensure that $p({\bf S},{\bf T})$ is properly
normalized.\footnote{Since our document pairs are always observed, we can safely
ignore the stopping cost $p(\epsilon, \epsilon)$ by assuming it to be some small
constant.} In this work, we will use a single parameter to weight these pairs:

\begin{align*}
p(S,T) &=& \lambda p_{Model1}(T|S) p_{LM}(S)\\
p(S,\epsilon) &=& \frac{1-\lambda}{2} p_{LM}(S)\\
p(\epsilon,T) &=& \frac{1-\lambda}{2} p_{LM}(T)
\end{align*}

$p_{Model1}$ and $p_{LM}$ refer to the IBM Model 1 and a unigram language model,
respectively. The parameter $\lambda$ roughly controls how eager the model
is to label sentence pairs as parallel. This can be set based on some prior
knowledge about the corpus.
\remove{
We will explore additional methods for setting
$\lambda$ in Section \ref{sec:extensions}.
}
$p_{Model1}$ is given by the
following equation from \citep{Brown93}:

\begin{equation}
p(T|S) = p\left(|T|\big||S|\right) \frac{1}{|S|^{|T|}}
\prod_{j=1}^{|T|} \sum_{i=1}^{|S|} p(t_j|s_i)
\end{equation}

For simplicity, we assume the source sentence $S$ contains the null word. The
term $\frac{1}{|S|^{|T|}}$ is the uniform alignment probability. The
length distribution, $p\left(|T|\big||S|\right)$, was originally described as a uniform distribution
over a large finite set of lengths. Since Model 1 is usually applied to parallel
corpora with observed sentence alignments, and the goal of using Model 1 is to
find word translation probabilities ($p(t|s)$), it is unnecessary to find an
accurate model of sentence length. However, when the sentence alignments are
being learned, it is important to have an accurate model of the length of the
target sentence given the source sentence. In this work, we use a Poisson
distribution to model the target sentence length, following \citet{Moore02}.

The probability for generating sentences monolingually, $p_{LM}(S)$, is a
unigram model estimated from the source language documents in the corpus.
Similarly, $p_{LM}(T)$ is estimated form the target language documents. While a
higher order language model could be learned, we use a unigram model to more
closely match IBM Model 1, which can be thought of as a mixture of unigram
models (one for each source word and one for the null word) that generate the
target sentence. We also use a Poisson distribution to model the lengths of 
monolingually generated sentences, rather than generating a special
end-of-sentence token.

\section{Data Collection}
\label{sec:data}
In order to evaluate the unsupervised sentence alignment model that we are
proposing, we must have bilingual document pairs with an annotated sentence
alignment. While existing parallel corpora may be used for this, the document
pairs in these corpora are highly parallel and would not resemble the alignments
found in Wikipedia articles on the same topic, or comparable news articles. We
will instead annotate comparable document pairs with their sentence alignment
using Amazon's Mechanical Turk (MTurk). 

\subsection{Mechanical Turk}

MTurk is an online marketplace where people may post collections of tasks
that workers may choose to complete for small amounts of money. These tasks are
referred to as Human Intelligence Tasks (or HITs) because they are intended to
be easy for humans to complete but difficult to automate. Examples of HITs
include the identification of offensive images, moderation of forum posts or
blog comments, and finding the contact information of a business. The workers on
MTurk are referred to as ``Turkers''.
MTurk has also been used for several natural language tasks \citep{Snow08},
including the evaluation of machine translation output \citep{Callison-Burch09}
and even translation itself \citep{Zaidan11}. The greatest concern when using
MTurk for annotation is ensuring that the results are reliable.

There are many ways in which sentence alignment of bilingual comparable
documents could be organized into HITs on MTurk. The simplest way would be to
take all possible sentence pairs in the document pair, and ask the Turkers to decide
whether or not they are parallel. Unfortunately, this will result in far too
many tasks to be affordable, as some Wikipedia articles have over a thousand
sentences. In order to cut down on the number of tasks, we applied pruning to
the candidate sentence pairs.

\subsection{Pruning and Data Selection}
\label{sec:turk_data}
Our pruning strategy is roughly based on that of \citet{Munteanu05}. Sentence
pairs are filtered by two criteria. {\bf Length ratio:} The ratio between the
lengths (in words) of the two sentences must be below a threshold in each
direction. {\bf Coverage:} The percentage of target words $t$ which either have an
exact string match with a source word, or have $p(t|s)$ (under IBM Model 1)
greater than a threshold for some $s$ in the source sentence. We obtain the
Model 1 probabilities by training on existing parallel data and bilingual
dictionary entries for the language pair. Coverage is computed on both the
source and target sentences, and a sentence pair is filtered if the average
coverage falls below a threshold.

This pruning strategy requires three thresholds to be set: a maximum length
ratio, a minimum average source/target coverage, and a minimum Model 1
probability for determining whether or not a word is covered. We tune these
thresholds on existing parallel data to ensure that the filter has high recall
(90\%) while still removing many non-parallel sentence pairs. For our
Urdu/English experiments, the thresholds we used were $2.5$ for the maximum
length ratio, $0.01$ for the minimum average coverage, and $0.575$ for the Model 1
word coverage threshold. We take our parallel data for training Model 1
parameters from the NIST MT09 Urdu-English training set and the
bilingual dictionaries and sentences gathered by \citet{Post12}.

In addition to pruning sentence pairs which are not likely parallel, we also
remove any pairs containing sentences with less than five tokens. Wikipedia
articles include section headings lists of names (such as an actor's
filmography), and links to other articles or external websites. Since our goal
is to find parallel sentences, we do not ask Turkers to annotate these very
short segments.

Since we are not asking Turkers to annotate all possible sentence pairs from an
article pair, evaluation becomes more difficult. We will discuss how we use our
partial annotation in Section \ref{sec:partial}.

\subsection{Task Design}
\label{sec:design}
Our strategy for designing the HITs on MTurk was to give the user an Urdu
sentence and a list of up to ten English sentences. The Turker is asked to
select which of the English sentences is parallel to the Urdu sentence, or
select ``None of the above'' if none of the English sentences are parallel. We
also ask if the sentence pair they find is a partial or full match, and give
some examples of each in the instructions. Figure \ref{fig:alignment_hit} shows
an example of one of these questions.

\begin{figure}[ht]
\begin{center}
\includegraphics[scale=0.5]{images/turk_hit.png}
\caption{The MTurk annotation interface for finding Urdu-English parallel sentences.}
\label{fig:alignment_hit}
\end{center}
\end{figure}

Our method of pruning potential sentence pairs may leave us with more than ten
candidate English sentences for some Urdu sentences. When this happens, we make
additional questions about these Urdu sentences to ensure all candidate pairs
are accounted for in the annotation. 

In each HIT, we ask the Turkers to annotate up to ten Urdu sentences with their
English counterpart (if any), including two control questions with sentences
taken from the parallel data described in Section \ref{sec:turk_data}. There is
one positive and one negative control in each HIT. We also request that each HIT
be done by three Turkers.

\subsection{Data Collection Results}
\label{sec:turk_results}
In our first large-scale experiment, we took 92 Urdu-English article
pairs, applied our filters as described in Section \ref{sec:turk_data}, and
uploaded our task to MTurk. While there were over 8 million possible sentence
pairs in these articles before pruning, we ended up with $785,000$ sentence
pairs to be annotated at a total cost of $\$726.80$ (this cost includes the duplicate
annotations).

Agreement among the Turkers was high ($\kappa = 0.84$). While the most common
answer was ``None of the above'', there were a substantial number of Urdu
sentences which the Turkers found some English counterpart for. For $21.4\%$ of Urdu
sentences, at least one Turker found one of the English sentences to be
parallel, and in $44.8\%$ of Urdu sentences, at least two Turkers identified a match.

% thesis-only: Using this as a way to get parallel data

\subsection{Evaluation Using Partial Alignments}
\label{sec:partial}
When we evaluate our sentence pair alignment model, we would like to compute the
precision and recall of the proposed sentence alignments. However, since we
prune many possible sentence pairs before asking the Turkers for annotation, we
cannot be sure whether or not some sentence pairs are parallel. In this section,
we will outline a scheme for evaluating sentence alignments using our partially
annotated data.

Our primary intrinsic evalutaion metric is alignment F-measure on sentence
alignments. This metric could also be seen as F-measure on a parallel sentence
pair retrieval task. Let $T$ be the set of true positives (sentence pairs that
are truly parallel), and $P$ be the set of predicted positives (sentence pairs
identified by our model as parallel). Precision, recall, and F-measure are
defined as follows:

\begin{align*}
\mbox{Precison}&=& \frac{|T \cup P|}{|P|}\\
\mbox{Recall}&=& \frac{|T \cup P|}{|T|}\\
\mbox{F-measure}&=& \frac{2 \cdot \mbox{Precision} \cdot
\mbox{Recall}}{\mbox{Precision} + \mbox{Recall}}
\end{align*}

When our document pairs are only partially annotated, we will used modified
definitions of precision, recall, and F-measure. Let $U$ be the set of sentence
pairs which were not annotated as parallel or non-parallel.

\begin{align*}
\mbox{Precison}&=& \frac{|T \cup P|}{|P \setminus U|}\\
\mbox{Recall}&=& \frac{|T \cup P|}{|T|}\\
\mbox{F-measure}&=& \frac{2 \cdot \mbox{Precision} \cdot
\mbox{Recall}}{\mbox{Precision} + \mbox{Recall}}
\end{align*}

Since $T$ and $U$ are disjoint, only the definition of precision needs to
be modified. 

Given the annotations we gathered from MTurk, it is possible to define $U$ in
multiple ways. The most conservative method would be to take $U$ to be all
sentence pairs not presented to the Turkers. However, if we make the assumption
that sentence alignments of the document pairs are $1:1$, then when a Turker
annotates a sentence pair $(S, T)$ as parallel, it follows that all $(S, T')$
pairs with $T' \neq T$ and $(S', T)$ pairs with $S' \neq S$ are not parallel.
Since the alignments we found were mostly $1:1$, we decided to go with this
option.\footnote{There were a small number of alignments which were not $1:1$,
most of which were image captions.} Figure \ref{fig:partial_align} illustrates
this method.

\begin{figure}
\begin{center}
\includegraphics[scale=0.5]{images/partial_alignment.png}
\caption{A partial alignment grid for a comparable document pair. The shaded
cells in the grid represent the sentence pairs which were presented to the
Turkers for annotation. A filled circle indicates the Turker found the sentence
pair to be parallel, and an empty circle means the pair is not parallel. The
dashed circles represent the sentence pairs we infer to be non-parallel by
assuming the sentence alignments are $1:1$.}
\label{fig:partial_align}
\end{center}
\end{figure}

\subsection{Alternate Evaluation Strategies}
% thesis only
Section \ref{sec:partial} describes a method for using Turkers' partial
annotation of a document pair's sentence alignment for intrinsic evaluation of a
sentence alignment model. In this section, we will explore other
strategies for using the Turkers' output for evaluation.

In our MTurk task setup (see Section \ref{sec:design}), we collect redundant
annotations for each HIT. While this was done primarily for quality control, and
it is more convenient to use a single judgement for each sentence pair, we can
perform a more fine-grained analysis by looking at the individual Turkers'
judgements. Also, we gave the option of labeling a parallel sentence pair as a
``partial'' or ``full'' match.

\NoteJS{TODO: For the semi-supervised experiments, we want to treat sentence
pairs that any annotator marked as a match as a true positive. We could also
measure the inter-annotator agreement of our system against the Turkers.}

\section{Experiments}
\label{sec:unsup_experiments}
Our first set of experiments uses a semi-supervised setting. We have both
parallel sentences (labeled data) and comparable document pairs (unlabeled data),
and learn our model's parameters from both of these resources.

Our parallel corpus is taken from the NIST MT09 Urdu-English training set
and the bilingual dictionaries and sentences gathered by
\citet{Post12}.\footnote{This is the same parallel corpus used to create the
sentence pair filters used in collecting the annotated sentence alignments.}
The parallel sentences from this corpus are treated as single sentence document
pairs. Alternatively, the entire training set could be seen as a single document
pair whose sentence alignment lies completely on the diagonal. The model
described in \ref{sec:alignment} does not differentiate between these two ways
of viewing the corpus. In either case, learning from the parallel sentences is
identical to IBM Model 1 training.

The comparable document pairs are a subset of the Wikipedia article pairs that
we annotated using MTurk as described in Section \ref{sec:data}. $60\%$ of this
data was taken as a development set. The remaining $40\%$ of the annotated
document pairs was split into two equal sized test sets.\footnote{This split was
done in order to have training, development, and test sets for supervised
sentence aligment models.}

In the following experiments the setup is as follows: We initialize our
parameters by running five iterations of EM on the parallel sentences from our
labeled data. Then we run several iterations of EM on both the labeled data and
unlabeled data, measuring performance after each iteration.

\subsection{Intrinsic Results}

\subsection{Extrinsic Results}
\NoteJS{This experiments section takes the new strategy where we start with lots
of supervision and then see how well we do as we remove data. Hopefully it will
end with good unsupervised results.}
The ultimate goal of parallel sentence extraction is to improve the quality of
end-to-end machine translation. In this section we will measure MT performance
before and after the extracted parallel sentences are added to the training
data.

\subsubsection{Corpora}
\begin{table*}
\small
\begin{center}
\begin{tabular}{|c|c|c|c|c|c|c|c|}
\hline
French & German & Polish & Italian & Dutch & Portuguese & Spanish & Japanese \\
496K & 488K & 384K & 380K & 357K & 323K & 311K & 252K\\
\hline
Russian & Swedish & Finnish & Chinese & Norwegian & Volap\"{u}k & Catalan & Czech \\
232K & 197K & 146K & 142K & 141K & 106K & 103K & 87K\\
\hline
\end{tabular}
\end{center}
\caption{The sizes of corpora in tokens and sentences for the Spanish-English
condition.}
\label{table:esen_corpora}
\end{table*}

\section{Conclusions}


\setcounter{chapter}{3}
\chapter{Previous Work}
\label{chap:related_work}
There have been several techniques developed for finding parallel data in a
wide range of multilingual corpora. 
Here we will review previous work in this area in
order to compare with the work we will present in later chapters. We will start
with the earliest work---the alignment of manually translated corpora.
\adlbr{Rather than think of this as a compendium of related work, 
it might be useful to think of a more goal-oriented chapter: setting
the stage for the rest of your dissertation. What did we know or
not know at the outset?}


% Sub-sentence alignment?
\remove{
\section{Temporary: Summaries}
This is a temporary section containing summaries of related papers. These will
be integrated into the above sections.

\citet{Dagan93}:
\citet{Chen00}: (PTMiner) Nothing special done for aligning the sentences in the 
parallel document pairs, only a method for finding document pairs. The method of
\citet{Simard93} was used for sentence alignment.
\citet{Shi06}: (DOM alignment)
\citet{Abdul-Rauf09}:
\citet{Ambati10}:
\citet{Ture11}:
\citet{Ture12}:

% Fragment alignment (a few missing here, need Munteanu07 citation)
\subsection{Parallel Fragments}
In addition to finding full parallel sentences, some researchers have looked for
parallel fragments within sentence pairs.
\citet{Munteanu06}: Starts with standard methods for finding candidate sentence
pairs. Then, computes log-likelihood ratio based scores for pairs of words, and
does a greedy word alignment based on these scores. (sliding window desc.)
\citet{Quirk07}:
\citet{Riesa12}:

\subsection{MT with Multiple Data Sources}

\subsection{Maxent Modeling}
There are a few things related to Maxent modeling that are relevant to
the supervised alignment chapter.

1. Should include some basic Maxent modeling information/citations
2. A citation on training set imbalance, or training/test mismatch
3. Feature binning/regularization: regularization over feature networks
}


% TODO: Switch back to IEEE for final submission
%\bibliographystyle{IEEEtran}
\bibliographystyle{plainnat}
\bibliography{refs}

\end{document}
