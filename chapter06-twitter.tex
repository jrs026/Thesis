\chapter{Comparable Data on Twitter}
\label{chap:twitter}

Methods for mining comparable corpora often leverage any available annotation
which may assist in finding parallel data. 
Many Web mining methods find parallel documents through their hyperlinks or
URLs \citep{Nie99,Chen00,Resnik99,Resnik03,Smith13}. \citet{Munteanu05} make use of the
publication date of news articles to narrow down the set of document pairs to
consider. Wikipedia contains ``interwiki links'' which directly identify
articles on the same topic in different languages, and within the articles,
other forms of markup such as links or images can help identify which
sentences are parallel \citep{Smith10}.
%While some methods avoid use of any
%annotation and work only with the text of the documents, they tend to be
%computationaly prohibitive \citep{Uszkoreit10,Ture12}.

Twitter is an interesting case, as its ``documents'' (tweets) are only 140 characters
long, and usually a single sentence at most. Tweets also contain many
potentially useful annotations: the author, the time it was posted, the names of
other users mentioned, and hashtags, which are informal topic annotations
({\tt \#worldcup}, for example). None of these are direct indicators that tweets
are parallel or even topic-aligned, but they still are useful in narrowing
the search space. 

% Prev. Twitter work (double check)
Previous work on mining parallel data from Twitter has used a
manually created list of terms associated with a topic that has multilingual
coverage \citep{Jehl12}, or searched for parallel
data within individual tweets \citep{Ling13}. We are interested in a general
methods for finding parallel tweets which do not require any initial knowledge
of topics.

In order to make the best use of the annotations available on Twitter, we
characterize the different sources of parallel data. Bilingual users may post
the same content in two languages, or directly translate another tweet. In
addition to these intentional translations, we also find coincidental
translations, where users independently tweet the same content about some
current event. These different types of parallel data are found by leveraging
different signals from Twitter. For each of these signals, we would like to
answer a few questions: Which languages are represented? A monolingual author or
hashtag cannot contain any parallel data. Also, how likely are we to
find parallel data? The hashtag {\tt \#no} will likely contain both Spanish and
English data, but not much of it will be parallel.

We use these signals to align potentially parallel tweets, and apply a
supervised classifier \citep{Munteanu05} to determine which pairs are actually parallel.
To test the quality of this data, we include SMT experiments which compare a
system trained with baseline data against one with our mined data added. We
assemble a test set of parallel tweets from GlobalVoices\footnote{{\tt globalvoicesonline.org}}
and show an improvement of X BLEU.

\section{Parallel and Comparable Tweets}
The term ``comparable corpus'' describes a wide array of multilingual corpora with varying
degrees of correspondence. \citet{Fung04a} give a more fine-grained
categorization based on the degree of topic-alignment and whether or not the
corpus was created by direct translation. When searching for parallel data on
Twitter, we will make the distinction between tweet pairs that are intentionally
translated, and pairs that are coincidentally translations. For example, many
Twitter users tweet the same information in multiple languages, on one or more
feeds, which gives us intentionally translated text. Users may also
independently tweet about some current event, and coincidentally convey the same
information. Figure \ref{tab:ex_tweets} shows examples of each.

\begin{table*}
\begin{tabular}{|c|c|}
\hline
English & can u listen to my song please? \\
\hline
Spanish & puede escuchar mi canci\'{o}n? \\
\hline
\hline
English & RT @celtics : Kobe Bryant surpassed Michael Jordan as the all-time NBA
\#AllStar scorer tonight.\\
\hline
Spanish & \#KobeBryant m\'{a}s puntos en \#AllStar games de todos los tiempos. :')\\
\hline
\hline
English & @NiallOfficial I love you, and do not lose faith that someday answer
me x9.\\
\hline
Spanish & @NiallOfficial te amo, y no pierdo la fe de que algu ́n día me
respondas.\\
\hline
\end{tabular}
\caption{Examples of intentional and coincidental translations on Twitter. The
first pair of tweets came from the same author, the second contained the
same hashtag ({\tt \#AllStar}), and the third contained the same mention ({\tt
@NiallOfficial}).}
\label{tab:ex_tweets}
\end{table*}

Naturally, we would expect different annotations on Twitter to be useful when
searching for intentional or coincidental translations. Looking at the author of
the tweet is important when searching for intentional translations. Other forms
of annotation, such as mentions, hashtags, and URLs are useful for aligning tweets by
topic. One important signal for finding any type of translation is the timestamp - we
expect translations to be published within a short timespan after the original
tweet.

Since we can't afford to search over all possible pairs of tweets, we use these
signals to narrow down the search space.
Similar to \citet{Munteanu05}, we use a sliding window when considering
candidate pairs of parallel tweets. In addition to this time-based matching, we
also require tweets to match in either their author, mention, hashtag, or URL to be
considered parallel.
We apply language identification \citep{Bergsma12} on matching sets of tweets to
further filter potential pairs to those in the correct source/target languages.

Once we have a filtered set of candidate pairs, we use the supervised classifier
of \citet{Smith10}, trained on their annotated Wikipedia data, to determine which
pairs are parallel.

\section{Notes}

\subsection{Matching Dimensions}
We align tweets by five dimensions: author, hashtag, mention, URL, and time.
We never use time alone to find parallel tweets - we use it to
further filter tweets matched by another attribute.

When matching tweets on the author attribute, we are attempting to find examples
of authors translating their own tweets into one or more other languages. We
expect that the translations will follow within a few minutes of the original tweet,
so a large time window is not required. We can also narrow down the search for
parallel tweets by only considering authors who regularly tweet in multiple
languages.

Hashtags are user-created topic markers. They have become used for things not
traditionally considered topics, such as {\tt \#ff} (follow Friday), or {\tt
\#sorrynotsorry}. We expect to find parallel tweets only from hashtags which
are related to current events, such as awards ceremonies ({\tt \#goldenglobes}),
sporting events ({\tt \#allstars}), and elections ({\tt \#Obama2012}).
We expect hashtags whose usage spikes at the same time in multiple languages to
be good potential sources of parallel data.

Tweets containing the same mentions ({\tt @user}) are not expected to be
parallel unless the user being mentioned is some public figure which many users
are likely to talk about. Actors, atheletes, and politicians are good potential
candidates for having a large multi-lingual following. We look for users who
have a high follower count and frequent mentions in our sample of tweets.

URLs are one source of matching tweets that is potentially independent of time:
two users may see an article at different times and tweet summaries in different
languages. Tweets containing the same URL are relatively rare, so even though
this is a strong signal that the tweets convey the same information, we don't
find large amounts of parallel data from this attribute.

There are some other attributes included in tweets which we decided not to use
for matching. When a tweet is created as a reply, it contains a reference to the
original. It also automatically includes a mention of the user being replied to,
so reply information was left out as redundant. Other metadata associated with a
tweet, such as the number of favorites, are not useful in matching potentially
parallel tweets.
% TODO: Check the auto-mention in replies


\section{Results}
We use a collection of tweets from Twitter's sample stream from 2012 (X
tweets total) as our input data. Our extraction algorithm is as follows:

\begin{enumerate}
\item Filter tweets for length and content---they must include at least five
tokens containing alphabetical characters, and excluding Twitter markup \adlul{such as}\adlmp{Be concrete. Is this everything you ignore?}
mentions and hashtags.
\item Perform language identification on the remaining tweets, and discard those
\adlul{not in the source or target language}.\adlmp{Is there some threshold?}
\item Index the remaining tweets by their attributes and a timestamp. The time
stamp is created by taking the time the tweet was posted, converting it to an
integer number of minutes, and dividing by the length of the time window (30
minutes in our experiments). Each tweet indexed by its current timestamp and the
immediately preceding one to ensure that tweets near time window boundaries are
correctly matched.\NoteJS{This needs a diagram to be explained clearly}
\item Apply a supervised parallel sentence classifier \citep{Smith10} to source and target tweets
that match on any attribute.\adlmp{after wikipedia chap.}
\item Remove \adlul{duplicated sentences pairs}\adlmp{Are there many?} from the resulting parallel data.
\end{enumerate}

Table \ref{tab:data_results} shows how much parallel data we find
for Spanish-English.

\begin{table*}[ht]
\begin{center}
\begin{tabular}{|c||c|c|}
\hline
Language & Segments & Tokens \\
\hline
\hline
English & 64,985 & 1,210,051 \\
\hline
Spanish & 64,985 & 1,158,787 \\
\hline
\end{tabular}
\end{center}
\caption{The amount of English-Spanish parallel data mined from our sample of
tweets from 2012. While it is possible that a single tweet may contain more than
one sentence, we are treating each tweet as its own segment.}
\label{tab:tweet_data_results}
\end{table*}

\section{Conclusions}
