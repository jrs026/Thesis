\chapter{Examining the Usefulness of Comparable Corpora}\footnote{This chapter
continues the work done with my team at the 2012 Machine Translation Marathon.
I would like to thank Philipp Koehn for performing SMT experiments, and my
teammates Herve Saint-Amand and Magdalena Plamada for both coding and analysis
of the parallel data.}
\label{chap:data}

Comparable corpora are multilingual collections of documents which are not
strictly parallel, but may contain some parallel data. This includes everything
from multilingual news feeds, which may not even be topic-aligned, to Wikipedia,
which has fine-grained topic alignment across languages.
This chapter will examine the feasibility of extracting large amounts of
parallel data from comparable corpora, with and without existing resources, and
analyze the effect this data has on end-to-end SMT performance.
Two comparable corpora are used here:
CommonCrawl, a publicly available crawl of the entire Web, and Wikipedia, an
online collaborative encyclopedia.
We conduct an extensive empirical exploration of the mined data, 
demonstrating coverage in a wide variety of languages and domains (\textsection\ref{sec:analysis}). Even 
without extensive pre-processing, the data improves translation performance
on strong baseline news translation systems in five different language pairs
(\textsection\ref{sec:experiments}).
On general domain and speech translation tasks where test conditions substantially
differ from standard government and news training text, web-mined training
data improves performance substantially, resulting in improvements of up
to 1.5 BLEU on standard test sets, and 5 BLEU on test sets outside of the news
domain.

\section{CommonCrawl}
A promising source of parallel data is the Web, as many websites are presented
in multiple languages. Researchers have been exploring ways to mine the Web for
parallel data for over a decade \citep{Resnik99,Nie99,Chen00}.
One major challenge is access to 
the data - large companies such as Google regularly maintain a crawl of the
entire Web, but even storing that much data may not be possible on a
university's local cluster.
In this work we are interested in methods for mining parallel data which are
feasible for researchers in academia to use.
For this reason we look at the CommonCrawl corpus,
which is a publicly available crawl of the web created by the CommonCrawl
foundation.\footnote{{\tt commoncrawl.org}}
As a baseline approach to both document alignment and sentence alignment, we
apply the STRAND algorithm \citep{Resnik03} to this dataset.

The CommonCrawl corpus is hosted on Amazon's Simple Storage Service (S3). It can
be downloaded to a local cluster, but the transfer cost (roughly 10 cents per
gigabyte, leading to a total of $\$81,000$\footnote{{\tt http://aws.amazon.com/s3/pricing/}})
is prohibitive. However, the data can be
accessed freely from Amazon's Elastic Compute Cloud (EC2) or Elastic MapReduce
(EMR) services, which charge modest fees for compute time, but allow free access
to data which is alread on S3. In our pipeline, we perform the first step of identifying
candidate document pairs using Amazon EMR, download the resulting document
pairs, and perform the remaining steps on our local cluster. We chose EMR
because our candidate matching strategy fit naturally into the Map-Reduce
framework \citep{Dean04}.

\subsection{STRAND Pipeline}

The following is the pipeline we use for our STRAND \citep{Resnik03} baseline:
\NoteJS{TODO: STRAND diagram}

\begin{enumerate}
\item {\em Candidate pair selection:} Retrieve candidate document pairs from the
CommonCrawl corpus
\item {\em Structural Filtering:}
  \begin{enumerate}
  \item Convert the HTML of each document into a sequence of start tags, end tags,
  and text chunks
  \item Align the linearized HTML of candidate document pairs
  \item Decide whether to accept or reject each pair based on features of the
  alignment
  \end{enumerate}
\item {\em Segmentation:} For each text chunk, perform sentence and word
segmentation
\item {\em Sentence Alignment:} For each aligned pair of text chunks, perform
the sentence alignment method of \citet{Gale93}
\item {\em Sentence Filtering:} Remove sentences which appear to be boilerplate
\label{enum:strand}
\end{enumerate}

\paragraph{Candidate Pair Selection}
Here we describe the Map-Reduce job which identifies candidate parallel
websites. We adopt a strategy similar to that of \citet{Resnik03} for finding
candidates in the Internet Archive.

Map-Reduce \citep{Dean04} is a framework for massively distributed parallel
programming over large amounts of data. It includes two main steps: a Map step
which creates key-value pairs from the input entries, and a Reduce step which
produces output from each key paired with all matching values.

The {\em mapper} operates on each website entry in the CommonCrawl data. It
scans the URL for some indicator of its language. Specifically, we check for:

\begin{enumerate}
\item Two/three letter language codes (ISO-639)
\item Language names in English and the language of origin
\end{enumerate}

If any of these codes or names are present in a URL and surrounded by non-alphanumeric
characters (for example: {\tt www.website.com/fr/}), this will be seen as a
potential match. The mapper will then output the following (key, value) pair:

\begin{itemize}
\item Key: {\tt www.website.com/*/}
\item Value: {\tt www.website.com/fr/}, French, (full website entry)
\end{itemize}

\NoteJS{TODO: detailed matching information, with code}

The {\em reducer} will then receive all websites mapped to the same
``language independent'' URL. If two or more websites match, the reducer will
output all matching document pairs, as long as they are not in the same
language, as determined by the language identifier from the URL.

This URL-based matching is a simple and inexpensive solution to the problem of
finding candidate document pairs. The mapper will discard most, and neither the
mapper nor the reducer do anything with the HTML of the documents aside from
reading and writing them. More sophisticated approaches have been used
\citep{Uszkoreit10,Ture12}, but they may be prohibitively expensive to run on
Amazon, and the focus of this work is to show that mining parallel data from the
entire Web can be affordable. \NoteJS{1250 hours for bitvector generation} 

\paragraph{Structural Filtering}
A major component of the STRAND system is the alignment of HTML documents. This
alignment is used to determine which document pairs are actually parallel, and
if they are, to align pairs of text blocks within the documents.

The first step of structural filtering is to linearize the HTML. This means
converting the DOM tree into a sequence of start tags, end tags, and chunks of
text. Some tags (those that may be often found within text, such as ``font'' and
``a'') are ignored during this step. Next, the tag/chunk sequences are aligned
using dynamic programming. The objective of the alignment is to maximize the
number of matching items.

Given this alignment, \citet{Resnik03} define a small set of features which
indicate the alignment quality. They annotated a set of document pairs as
parallel or non-parallel, and trained a classifier on this data. We also
annotated 101 Spanish-English document pairs in this way and trained a
maximum entropy classifier. However, even when using the best performing subset
of features, the classifier only performed as well as a naive classifier which
labeled every document pair as parallel, in both accuracy and F1. For this reason,
we excluded the classifier from our pipeline.

\paragraph{Segmentation}
The text chunks from the previous step may contain several sentences, so before
the sentence alignment step we must perform sentence segmentation. We use the
Punkt sentence breaker from NLTK \citep{nltk} to perform both sentence and word
segmentation on each text chunk.

\paragraph{Sentence Alignment}
For each aligned text chunk pair, we perform sentence alignment using the
algorithm of \citet{Gale93}.

\paragraph{Sentence Filtering}
Since we do not perform any boilerplate removal in earlier steps, there are many
sentence pairs produced by the pipeline which contain menu items or other bits
of text which are not useful to an SMT system. To remove this data, we prune
segment pairs unless both segments contain at least 5 tokens composed of
alphanumeric characters only, and end with punctuation. We also remove any
sentence pairs which are identical.

\subsection{Results}
\subsubsection{Intrinsic Evaluation}

\begin{table}[ht]
\begin{center}
\begin{tabular}{|c|c|}
\hline
Language & Precision \\
\hline
Spanish & 82\% \\
\hline
French & 81\% \\ 
\hline
German & 78\% \\
\hline
\end{tabular}
\end{center}
\caption{Precision on the extracted parallel data for Spanish, French, and
German (paired with English).}
\label{tab:intrinsic}
\end{table}

To evaluate the quality of the parallel data produced , we manually check a set of randomly selected 200 sentence
pairs for three language pairs. The
texts are very heterogeneous, covering several topical domains, such as:
tourism, advertising, technical specifications, finances, e-commerce or
medicine. 
For German-English, 78\% of the extracted data represent perfect translations, 4\% are
paraphrases of each other (convey a similar meaning, but cannot be used for SMT
training) and 18\% represent misalignments. Furthermore, 22\% of the true
positives are potentially machine translations, whereas in 13\% of the cases one
of the sentences contains an extra tail. As for the false positives, 13.5\% of
them have been caused by language identification errors, the remaining ones
representing failures in the alignment process. Similar tendencies have been
observed in the other data sets.
All in all, the precision of the mining process is on average 80\% for the
considered language pairs, as Table \ref{tab:intrinsic} shows.
This analysis suggests that language identification and SMT output
detection \citep{Venugopal11} may be useful additions to the pipeline.

\subsubsection{Extrinsic Evaluation}
We applied this baseline to the full 2009-2010 crawl in seven separate
chunks.\footnote{This was to avoid out-of-memory errors which occurred when
running on the full crawl in a single experiment.}
In Table \ref{tab:cc_mt} we show SMT
experiments before and after adding this data to a baseline.
In both the baseline and experiments with added data we include the target side of the mined
parallel data in the language model, to show the improvements are not just
coming from the additional monolingual data.
These results show substantial gains on top of an already strong SMT system.

\begin{table}[ht]
\begin{center}
\begin{tabular}{|r||r|r|r|r|r|}
\hline
& \textbf{FR-EN} & \textbf{EN-FR} & \textbf{ES-EN} & \textbf{EN-ES} & \textbf{EN-DE}\\
\hline
\textbf{Baseline} \
      & 29.88 & 28.50 & 32.80 & 32.83 & 16.61  \\
\hline
\textbf{+Web Data} \
      & 30.08 & 28.76 & 33.39 & 33.41 & 17.30  \\
\hline
\end{tabular}
\end{center}
\caption{BLEU scores for several language pairs before and after adding the mined
parallel data to a baseline system trained on data from WMT 12.}
\label{tab:cc_mt}
\end{table}

\section{Wikipedia}
Wikipedia\footnote{{\tt www.wikipedia.org}} is an online collaborative
encyclopedia available in over 200 languages. It is part of the larger WikiMedia
project, which includes other multilingual websites such as Wiktionary and
Wikinews. It is a promising source of parallel data due to the ``interwiki''
link structure. Each article has links to articles on the same topic in other
languages. These are occasionally direct translations, but for the most part
they are simply topic-aligned.

While this is of course included in the Web, the web mining techniques used in
general Web mining are not appropriate for this corpus. First, URL matching
techniques will not work for the majority of bilingual article pairs. Also, the
article pairs in Wikipedia are not often direct translations, so methods that
rely on HTML alignment, such as STRAND, are not appropriate. 
Since Wikipedia provides document alignment across languages via ``interwiki''
links, the document alignment step is unnecessary, though see \citet{Ture12} - the
authors ignore the interwiki link structure and align documents themselves.
While the links could be missing some article pairs, they are fairly well
maintained by Wikipedia contributors. \NoteJS{I should compare with his data if
possible.}

\subsection{Software}
Here we will describe the pipeline we use for mining parallel data from
Wikipedia, and show how each of the steps can be done using the
WikiDumpTools\footnote{TODO: Github URL} software package. We work with static
dumps of Wikipedia to avoid constant bandwidth usage and to ensure consistent
results across several experiments.

\begin{enumerate}
\item For each language we are working with, download the dump of all articles,
the interwiki link table, and the redirect table
\item Create indexed versions of each dump for quick random access
\item Create a list of article pairs using the interwiki link tables and
redirect tables for each language pair
\item Iterate through the list of article pairs, retrieve their Wikitext from
the indexed dump, and output plain text document pairs
\end{enumerate}

\paragraph{Downloading static dumps}
Database dumps of Wikipedia can be found at {\tt dumps.wikimedia.org}. They are
listed by language code, so ``enwiki'' is the English Wikipedia, ``eswiki'' is
the Spanish, etc. There are many different types of database dumps here, but the
ones we are interested in are the main articles, the interwiki link table, and
the redirect table. These files end in ``{\tt pages-articles.xml.bz2}'',
``{\tt langlinks.sql.gz}'', and ``{\tt redirect.sql.gz}'', respectively.

\paragraph{Indexing static dumps}
When iterating through the interwiki links, we need to quickly find articles in
the database dumps by title. In order to do this, we build an indexed dump,
where each entry in the index contains byte offsets for the Wikitext of an
article.

\begin{verbatim}
python wdtools.py --index-wiki (dump file) --output indexed-wiki
\end{verbatim}

The dump file should be the file that ended in ``{\tt pages-articles.xml.bz2}''
This command will create two files: {\tt indexed-wiki.index.gz}, which is the
index to the dump, and {\tt indexed-wiki.dump}, which is the uncompressed
Wikitext of all articles in the original dump. This file is uncompressed to
allow efficient random access.

\paragraph{Finding article pairs}
The interwiki links table contains the outgoing interwiki links for the articles
in each language. The redirect table contains a record of all redirect pages,
which are usually spelling variants. WikiDumpTools uses the interwiki link
tables and redirect tables from two languages to create a complete list of
article pairs matched across languages.\footnote{The redirect tables are needed
since the interwiki link can point to a redirect page.} Note that this list of
article pairs is, for the most part, parallel data itself, and we explore its use in
later chapters.

\begin{verbatim}
python wdtools.py --get-pairs (output file) --source (dump index),(interwiki
table),(redirect table) --target (dump index),(interwiki table),(redirect table)
\end{verbatim}

``Source'' and ``target'' here refer to the source and target languages. The
dump indexed is required because it contains a mapping from article IDs to
article titles, and the other tables make use of both.

\paragraph{Creating article pairs}

Using the list of article pairs, we can now output plain text document pairs.

\begin{verbatim}
python wdtools.py --output-pairs docs-out --source-dump (indexed dump)
--target-dump (indexed dump) --pair-list (list of article pairs)
\end{verbatim}

This will iterate through the list of article pairs, retrieve them from the
indexed dumps, remove Wikitext markup, and create lists of documents in the
source and target languages. Two files will be created: {\tt docs-out.source.gz}
and {\tt docs-out.target.gz}. These files are aligned document pairs, where
document boundaries are separated by empty lines. Note that no sentence breaking
or word tokenization is performed.

After the steps in this pipeline, comparable corpus mining techniques can be
used on the document pairs. Software for performing this step will be described
in later chapters.

\section{Comparison to Current Parallel Resources}
For several language pairs, there are parallel corpora already available. Some
notable examples include Europarl \citep{Koehn05}, the U.N. Corpus
\citep{MultiUN}. In addition to these
multilingual corpora, there are many parallel corpora for specific language
pairs. For the most part, the language pairs that received the most attention
are European languages, Arabic, and Chinese paired with English.

In this section, we will compare the amount of existing parallel data for many
language pairs with the amount of comparable data available from our two
sources, CommonCrawl and Wikipedia. This is meant to give an estimate of how
much parallel data we can expect to mine from these sources. To estimate the
amount of existing parallel data, we use OPUS, a freely available collection of
parallel data \citep{OPUS}. Table \ref{tab:opus_data} shows the amount of existing
parallel data for selected languages paired with English. \NoteJS{I will have to
double check these carefully, since there are a few corpora I know of that are
missing, and older versions of Europarl are used in the counts. The counts are
also dominated by the OpenSubtitles corpus, which people rarely use.}

\begin{table}
\begin{center}
\begin{tabular}{|r||r|r|r|r|r|}
\hline
& \textbf{BG} & \textbf{CS} & \textbf{DE} & \textbf{ES} & \textbf{FR} \\
\hline
Segments      & 18.5M  & 22.8M  & 8.3M   & 51.5M  & 39.7M  \\
\hline
Source Tokens & 137.6M & 153.7M & 92.1M  & 361.4M & 247.6M \\
\hline
Target Tokens & 163.7M & 187.9M & 100.9M & 331.0M & 235.7M \\
\hline
\hline
& \textbf{JA} & \textbf{KO} & \textbf{RU} & \textbf{UR} & \\
\hline
Segments      & 0.5M  & 0.1M  & 18.3M   & 1.0K  & \\
\hline
Source Tokens & 5.4M  & 3.8M  & 48.5M   & 2.6M  & \\
\hline
Target Tokens & 0.5M  & 0.4M  & 39.2M   & 10.0K & \\
\hline
\end{tabular}
\end{center}
\caption{The amount of parallel data available from OPUS \citep{OPUS} for each
language paired with English. Source tokens are counts of the foreign language
tokens, and target tokens are counts of the English language tokens.}
\label{tab:opus_data}
\end{table}

Table \ref{tab:cc_data} shows the amount of parallel data mined from CommonCrawl
on the same language pairs using the baseline system STRAND. Table
\ref{tab:wiki_data} shows an upper bound on the amount of data that could be
extracted from Wikipedia. The numbers shown are the sum of the number of tokens
of the smaller article in each bilingual article pair.

\begin{table}
\begin{center}
\begin{tabular}{|r||r|r|r|r|r|}
\hline
& \textbf{BG} & \textbf{CS} & \textbf{DE} & \textbf{ES} & \textbf{FR} \\
\hline
Segments      & 962K  & 886K  & 8.04M & 6.11M & 10.9M  \\
\hline
Source Tokens & 8.72M & 7.50M & 83.9M & 75.4M & 135M \\
\hline
Target Tokens & 8.53M & 7.95M & 88.4M & 68.8M & 121M \\
\hline
\hline
& \textbf{JA} & \textbf{KO} & \textbf{RU} & \textbf{UR} & \\
\hline
Segments      & 1.80M  & 787K   & 3.86M   & 59.7K & \\
\hline
Source Tokens & 9.59M  & 6.57M  & 36.6M   & 828K & \\
\hline
Target Tokens & 19.1M  & 7.42M  & 37.2M   & 723K & \\
\hline
\end{tabular}
\end{center}
\caption{The amount of parallel data mined from CommonCrawl for each
language paired with English. Source tokens are counts of the foreign language
tokens, and target tokens are counts of the English language tokens.}
\label{tab:cc_data}
\end{table}

\begin{table}
\begin{center}
\begin{tabular}{|r||r|r|r|r|r|}
\hline
& \textbf{BG} & \textbf{CS} & \textbf{DE} & \textbf{ES} & \textbf{FR} \\
\hline
Tokens (max)  & 19M  & 33M  & 176M & 148M & 174M  \\
\hline
\hline
& \textbf{JA} & \textbf{KO} & \textbf{RU} & \textbf{UR} &\\
\hline
Tokens (max)  & 29M & 22M & 107M & 1.9M & \\
\hline
\end{tabular}
\end{center}
\caption{An upper bound for the amount of parallel data from Wikipedia for each
language paired with English.}
\label{tab:wiki_data}
\end{table}

From the annotated Wikipedia article pairs collected by \citet{Smith10}, we can
estimate the relationship between the upper bounds shown in Table
\ref{tab:wiki_data} and the number of tokens of parallel data. For
Spanish-English, the ratio is $0.35$, for German-English it is $0.22$, and for
Bulgarian-English it is $0.18$. This is only giving us very coarse estimates of
the amount of parallel data available, but it gives us some idea of how much
of an improvement we can expect when adding the mined data to an existing SMT
system.
